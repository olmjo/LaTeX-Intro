\part{\LaTeX{} Miscellany}

\section{Tables}
\begin{itemize}
\item We can create tabular content in a \LaTeX{} document either within
  a float environment or not. We will describe what a float is in a moment.

\item Either way, tabular content like the following is created with
  the \texttt{tabular} environment.

\begin{center}
  \begin{tabular}{c c c}
    \hline
    \hline
    State             & Capital    & State Bird             \\
    \hline
    New Jersey        & Trenton    & Eastern Goldfinch      \\
    Connecticut       & Hartford   & American Robin         \\
    New York          & Albany     & Eastern Bluebird       \\
    Massachusetts     & Boston     & Black-capped Chickadee \\
    \hline
    \hline
  \end{tabular}
\end{center}


\item The \texttt{tabular} environment takes a mandatory argument that is
  enclosed in curly braces. In the case of the above table where each column is
  center justified, we'd use \verb! \begin{tabular}{c c c} \end{tabular}!. If we
  wanted the columns to be right-justified or left-justified we could use
  \texttt{r} or \texttt{l} in place of the three \verb=c='s.

\item If we wanted to add a vertical line on either the left or right or in-between two columns we'd use the  argument \verb!{| c | c | c |}!.

\item Horizontal lines are given by \verb!\hline!.

\item Tab breaks are caused by \verb!&! and line breaks by \verb!\\!.

\item The full code for the above table is:\\
  \lstinputlisting{./chapters/05_misc/table.tex}

\item There are much more complex environments than \texttt{tabular} that you
  will have to use once your tabular content grows. They are similar, but have
  their own documentation to help you through.

\end{itemize}

\section{Pictures}

\begin{itemize}

\item \LaTeX{} has the ability to draw arbitrary types of objects and schematics
  within a \LaTeX{} document in a native language. However, this is sometimes
  overkill. If necessary, you can learn about this later.

\item Download \texttt{Image1.pdf} and \texttt{Image2.png} from
  \url{http://github.com/olmjo/LaTeX-Intro/tree/master/extra}.

\item In the preamble of your working document, add the \texttt{graphicx}
  package (i.e.\ \verb!\usepackage{graphicx}!)

\item Where you'd like the picture to show up, type
  \verb!\includegraphics{./Image1}! without a file extension and be
  sure to use the appropriate path to the image file, relative to the
  \LaTeX{} document.

\item The \verb!scale! optional argument is useful to change dimensions on the
  fly without distorting the aspect ratio.

\item Try the following code in a \LaTeX{} document environment:

\begin{lstlisting}
  \begin{center}
    \fbox{
      \fbox{
        \includegraphics[scale=.5]{./extra/Image1}
      }
      \fbox{
        \includegraphics[width=3in,height=1in,angle=90]{./extra/Image2}
      }
    }
  \end{center}
\end{lstlisting}

\item The result is something like

  \begin{center}
    \fbox{
      \fbox{
        \includegraphics[scale=.5]{./extra/Image1}
      }
      \fbox{
        \includegraphics[width=3in,height=1in,angle=90]{./extra/Image2}
      }
    }
  \end{center}

\item However, this will work only if the images are located in a
  directory/folder called \verb=extra= which is in the same directory as your
  main \LaTeX{} file (as is the case with this document).

\item Notice that we didn't have to specify the file extension. Notice
  the order in which the \texttt{angle} and then
  \texttt{height}/\texttt{width} arguments are applied.

\item What does \verb!\fbox{}! do?


\end{itemize}

\section{Floats}

\begin{itemize}

\item Although we can create tables with the \texttt{tabular} environment and we
  use \verb!\includegraphics{}! to insert external image files, these commands
  are seldom placed inside a document without entering them in a special
  environment.

\item Typically, these kinds of content are placed in \textit{floats} which act
  like containers for tables and figures. \LaTeX{}, according to a set of rules,
  determines where these containers should be placed.

\item The ultimate placement depends on a number of factors: the content before
  the float, the content after the float, and the \LaTeX{} code within the
  float.

\item If nothing else, the big adjustment required by users placing content in
  floats is realizing that the placement of a table or figure is up to \LaTeX{}
  and that ``jury-rigging''/''jimmy-rigging''/''jerry-rigging'' the placement is
  ill-advised.

\item The float for tables is \verb!\begin{table}[htpb] \end{table}!. The
  optional \verb![thpb]!  argument provides \LaTeX with some instructions on
  where to place the table.

\item The float environment for figures is
  \verb!\begin{figure}[htpb] \end{figure}!. Notice, again, the optional
  argument.

\item In actuality, there need not be anything inside these float
  environments, or it could easily be regular \LaTeX{} markup.

\item \verb!\caption{}!, placed somewhere in the float, allows a title
  of the content to be placed and automatically numbered.

\item \verb!\label{}!, placed immediately after the \verb!\caption{}!
  command gives an identified to the object by which it can be
  referred for directing readers to Figure 1 or Table 4 without
  hard-coding the float order.

\item Try: \\
\begin{lstlisting}
  \begin{figure}
    \begin{center}
      \fbox{
        \includegraphics[width=3in,height=1in,angle=120]{./Image2}
      }
      \caption{A Hero of a Man} \label{f:Riker}
    \end{center}
  \end{figure}
\end{lstlisting}

\item Now, see Figure \ref{f:Riker} on page \pageref{f:Riker} to view the
  output. We were able to reference the figure number automatically using
  \verb!\ref{f:Riker}! which matches our \verb!\label!. We reference the page
  number automatically with \verb!\pageref{}!.
  \\

\begin{figure}
  \begin{center}
    \fbox{
      \includegraphics[width=3in,height=1in,angle=120]{./extra/Image2}
    }
    \caption{A Hero of a Man} \label{f:Riker}
  \end{center}
\end{figure}

\end{itemize}

\section*{More Math Environments}

\begin{itemize}

\item There are a number of math environments that become useful when one is
  typesetting mathematical notation beyond very basic experessions.

\item As in a previous tutorial, add
  \verb!\usepackage{amsmath, amsthm, amssymb, amsfonts}! to the preamble if not
  already there.

\end{itemize}

\subsection*{Fractions}

There are instances where $3/4$ would look better as $\frac{3}{4}$ and this
works equally well for longer expressions $$ \frac{\tan\left(\cos\left(\sin (X)
    \right)\right)}{\int_{\mathbb{R}} f(x) ~ dx}.$$ It is the command
\verb!\frac{}{}! which provides this. The numerator is the first argument and
the denominator is the second.

So, \verb=$\frac{a}{b}$= gives $\frac{a}{b}$.

\subsection*{Equation}

The equation environment is a very common way to typeset equations
when reference numbers are being used. So, for example,
\verb!\begin{equation} 4=x^2 \end{equation}! gives
\begin{equation}
  4=x^2.
\end{equation}

Now, it is not necessary to place the equation environment in a math
mode. In this sense, the math mode is implied. There is also a variant
such that \verb!\begin{equation*} 4=x^2 \end{equation*}! suppresses
the equation numbering,
\begin{equation*}
  4=x^2.
\end{equation*}

Equation numbers can be labeled and referenced as was done in the figure
environment. This is a single equation environment and if you try to enter line
breaks such that you could force another line, it will fail. In that case, I
find the \texttt{align} approach being the easiest because it is flexible.

\subsection*{Align}

The \texttt{align} environment allows you to enter multiple lines and
include alignment stops. There is a un-numbered version,
\texttt{align*}, too. The code

\begin{lstlisting}
\begin{align}
   \sum_{x=1}^{4} \frac{1}{x} &
   &= \frac{1}{1} + \frac{1}{2}  + \frac{1}{3} + \frac{1}{4} & \\
   &= \frac{25}{12} &\\
   & &>2
\end{align}
\end{lstlisting}

produces a three line \texttt{align} environment.
\begin{align}
   \sum_{x=1}^{4} \frac{1}{x}
   &= \frac{1}{1} + \frac{1}{2}  + \frac{1}{3} + \frac{1}{4} \\
   &= \frac{25}{12} \\
   &>2
\end{align}

Notice that the numbering is cumulative and that both \verb!\label{}!  and
\verb!ref{}! work identically here.


\subsection*{Array}

The \texttt{array} environment provides a unified way of representing vectors
and matrices. The environment begins in the standard
\verb!\begin{array}{c c} \end{array}! way, but the mandatory \verb!{c c}! argument
specifies there are two center-justified columns.

Changes to this argument proceed identically to the argument in the creation of
tables. One important point is that the array environment does not create its
own math mode environment, so we must put it inside one when we use it.

We get

\[
\left[
  \begin{array}{cc}
    1 & 0 \\
    0 & 1
  \end{array}
\right],
\]

the two dimensional identity matrix, from

\begin{lstlisting}
\[
\left[
  \begin{array}{cc}
    1 & 0 \\
    0 & 1
  \end{array}
\right],
\]
\end{lstlisting}

Notice the comma in the markup code. It must be in the display math environment
so that it is placed adjacent to the matrix and not in the text. By changing the
number of rows, columns, and the delimiters, most matrix-like objects can be
represented with this environment.

\subsection*{Cases}

The \texttt{cases} environment is both extremely useful, but quite
narrow in application. Although it is designed to be used only to
represent piece-wise functions, it is a marked improvement over the
alternative which would be to ``hack'' the \texttt{array}
environment. Like the \texttt{array} environment, though,
\texttt{cases} must be used within math mode. So,

\[
\mathbf{1}_{\mathcal{X}}(x) =
\begin{cases}
  1, & x \in \mathcal{X} \\
  0, & \textrm{otherwise}
\end{cases}
\]
is the result of the code

\begin{lstlisting}
\[
\mathbf{1}_{\mathcal{X}}(x) =
\begin{cases}
  1, & x \in \mathcal{X} \\
  0, & \textrm{otherwise}
\end{cases}.
\]
\end{lstlisting}

Notice how I use whitespace and line breaks to organize the code although
\LaTeX{} won't interpret it. If you add many spaces between the $x$ and the
$\in$ symbol, it doesn't actually affect the display of the function.

\section{Words of Wisdom}

\begin{itemize}

\item It is considered good practice to keep text file contents within the first
  80 characters. This may seem weird or hard, but this, along with use of the
  \texttt{\%} and comments will make the input file more human-readable.
    \item As you learn \LaTeX{} don't worry about trying to make
      \LaTeX{} look a certain way. Tell \LaTeX{} about your content
      and its structure. Let \LaTeX{} worry about the details of
      appearance.

    \item Google is your friend.

    \item Comment out error-laden parts of code. Add things back in one at a
      time until you've identified the source of your syntactical mistakes.

    \item Let your text editor help you. Good ones highlight your \LaTeX{}
      markup according to rules. If the rules are broken, the highlighting will
      appear other it should and this is a visual cue that something is
      wrong. For example, many text editors will make fold type like that with
      \verb=\textbf{}= bold. If you see entire paragraphs displayed in your text
      editor in a bold face, this may be an indication that you forgot to end
      the region that should be bold.

    \item Because \LaTeX{} seldom interprets whitespace in too generous a way,
      use it as an organizational tool.

\end{itemize}


%%% Local Variables:
%%% mode: latex
%%% TeX-master: "../../tutorial"
%%% End: