\part{Basic Content in LaTeX{}}

\section{\LaTeX's Interpretation of Plain Text}

\begin{itemize}
\item The entire \texttt{.tex} file you write will contain only characters
  on your keyboard. That means, somehow, your using characters on your
  keyboard you need to represent: a, A, $\alpha$, $\mathbb{A}$, \`a,
  \texttt{A}, $_a$, and \"A.
  
\item Your entire document will be composed of the following characters
  (including documents utilizing \staveVI): \\
  \ovalbox{
    \begin{minipage}{\linewidth}
\begin{verbatim}
a b c d e f g h i j k l m n o p q r s t u v w x y z
A B C D E F G H I J K L M N O P Q R S T U V W X Y Z
0 1 2 3 4  6 7 8 9 
! @ # $ % ^ & * ( ) _ +
{ } [ ] | \ l ; : ' " < , > . ? /
\end{verbatim}
    \end{minipage}
  }
\end{itemize}

\subsection*{Special Characters}
\begin{itemize}
\item The following characters are reserved or \textit{special}:
\begin{verbatim}
% # $ ^ & _ { } ~ \
\end{verbatim}
  
\item If you type ``\texttt{\%}'' into WinEdt and \texttt{pdflatex} your file, you
  will not get a percent sign.
  
\item In order to generate these symbols you need to use special
  commands. Although detailed descriptions of these symbols is beyond
  the purpose of this section, accept the following brief comments:\\
  \ovalbox{
    \parbox{\linewidth}{
      \begin{description}
      \item[\texttt{\%}:] used for including comments and preventing \LaTeX ~from
        interpreting file contents
      \item[\texttt{\#}:] used to define \LaTeX ~commands (or macros) 
      \item[\texttt{\$}:] used to input \LaTeX ~math
      \item[\texttt{\textasciicircum{}}:] used for superscripts in math mode
      \item[\texttt{\&}:] used for ``tab stops'' and alignment and tables
      \item[\texttt{\_}:] used for subscripts in math mode
      \item[\texttt{\{ \}}:] used to issue arguments to \LaTeX ~commands
      \item[\texttt{\~{}}:] used to represent a special kind of space
      \item[\texttt{\textbackslash}:] used to start every \LaTeX ~command
      \end{description}
    }
  }
\end{itemize}

\subsection*{Commands}

\begin{itemize}

\item \par Using \LaTeX ~ efficiently often requires one be familiar with
  various \LaTeX commands (or macros). In general, this comes only with
  practice. Commands start with a ``\texttt{\textbackslash}'' and they
  are case-sensitive. Commands can be used to generate particular glyphs
  or alter the content in some way.
  
\item \par One command is ``\texttt{\textbackslash LaTeX\{\}}'' which
  produces ``\LaTeX''. Similarly, ``\texttt{\textbackslash
    textbf\{election\}}'' produces ``\textbf{election}''.
  
\item \par You'll never learn a significant proportion of all the \LaTeX~
  commands that people use. However, you'll eventually memorize the ones
  you use over and over and know how to look up the rest.
  
\item \par The mandatory arguments to a macro are passed inside curly braces
  and the option arguments to a macro are passed inside square
  brackets. The result is something like 
\begin{verbatim}
\command[optional1=1, optional2=2]{mandatory=Always}.
\end{verbatim}
  
\end{itemize}

\subsection*{Whitespace and Spacing} 

\begin{itemize}
\item 2 or more carriage returns,
  ``\texttt{\textbackslash\textbackslash}'', and
  ``\texttt{\textbackslash linebreak}'' break lines
\item ``\texttt{\~}'' is a non-breaking space
\item ``\texttt{\textbackslash par}'' will create a new paragraph for the succeeding
  text, regardless of surrounding whitespace
\item ``\texttt{linebreak}'' and ``\texttt{pagebreak}'' are
  appropriately named
\item multiple consecutive spaces are interpreted as one
\end{itemize}

So, \\
\ovalbox{
  \begin{minipage}{\linewidth}
\begin{verbatim}
Keith         Poole \\



      once 
           proclaimed \\ at~a~breakfast
here in Rochester,\par ``I am unfireable!''
\end{verbatim}
  \end{minipage}
}
becomes \\
\ovalbox{
  \begin{minipage}{\linewidth}
    Keith Poole \\



    once 
    proclaimed \\ at~a~breakfast
    here in Rochester,\par ``I am unfireable!''
  \end{minipage}
}

\section{\texttt{.tex} Input File Structure}

\begin{itemize}

\item \par \LaTeX~\texttt{.tex} files have a particular structure. If the file you attempt to
  compile has an error, your document will either fail to be produced or
  it will not be compiled in accordance with your intentions.

\item \par First, you must declare the document class: \\
  \ovalbox{
    \begin{minipage}{\linewidth}
\begin{verbatim}
\documentclass[letterpaper, 10pt]{article}
\end{verbatim}
    \end{minipage}
  } 
  or \\
  \ovalbox{
    \begin{minipage}{\linewidth}
\begin{verbatim}
\documentclass[12pt]{letter}
\end{verbatim} 
    \end{minipage}
  }
  and this is part of the \textit{preamble}. For your purposes,
  ``article'' is sufficient for the time being.

\item \par Second, the remainder of the preamble contains explicit calls to
  outside packages if you require their functionality, the creation of
  new macros, and setting document properties. After the
  \texttt{\textbackslash documentclass} command we might see the
  following lines:\\
  \ovalbox{
    \begin{minipage}{\linewidth}
\begin{verbatim}
\usepackage{fancyhdr}

\pagestyle{fancy}
\cfoot{\thepage}
\title{A Document}
\end{verbatim}
    \end{minipage}
  } which provides the functionality from the ``\texttt{fancyhdr}''
  package, sets the ``\texttt{pagestyle}'' to ``fancy'', places the page
  number in the center of the footer, and sets the title to ``A
  Document''. This is also part of the \textit{preamble}.

\item Third, the main content is typed within the ``\texttt{document}''
  environment, as in:\\
  \noindent \ovalbox{
    \begin{minipage}{\linewidth}
\begin{verbatim}
\begin{document}

%% place content here

\end{document}
\end{verbatim}
    \end{minipage}
  }

\item Lastly, any characters occurring after the close of the
  environment will be ignored by \LaTeX. This includes special
  characters and commands.

\end{itemize}

In its entirety, we have something like \\
\ovalbox{
  \begin{minipage}{\linewidth}
\begin{verbatim}
\documentclass[12pt]{article}

\title{An Essay on Rob}

\begin{document}

\maketitle

Rob is a cheery chap whose trumpeted lips are sometimes chapped.

\end{document}

$ breaking \ rules # $ without }{ effect
\end{verbatim}
  \end{minipage}
} \\

\marginpar{
  Go ahead and compile this document after you type it out.
}

\section{Output File Structure}
\subsection*{Headings}

\begin{itemize}
\item \LaTeX~accepts the definition of a document hierarchy and displays
  headings appropriately. The \textit{article} class accepts the
  following headings whose order indicates how far down they are in the
  order: \texttt{\textbackslash part\{\}}, \texttt{\textbackslash
    section\{\}}, \texttt{\textbackslash subsection\{\}},
  \texttt{\textbackslash subsubsection\{\}}, \texttt{\textbackslash
    paragraph\{\}}, and \texttt{\textbackslash subparagraph\{\}}. If an
  ``\texttt{*}'' is placed after the name as in ``\texttt{\textbackslash
    part*\{\}}'' then the heading will not be numbered. Otherwise it
  will. The one mandatory argument is the text of the heading.

\item Any headings after the \texttt{\textbackslash appendix} command will
  be altered to reflect that they do not belong to the main body of
  text. \marginpar{Add a \texttt{section} and a \texttt{subsection*}
    heading to your document. Name them after your two favorite
    colors. Compile the code.}
  
\end{itemize}

\section{Environments}

\begin{itemize}
\item All of the content that will be displayed in your final document is
  contained in at least one environments are denoted by the
  \texttt{\textbackslash begin\{env\}} and \texttt{\textbackslash
    end\{env\}} commands. We have already seen the \texttt{document}
  environment.
  
\item We will discuss various environments (and there are many) during the
  remainder of the week, but the basics are all the same. If you begin
  an environment, you must end it. When nesting environments, the most
  recently opened environment must be closed before an earlier
  environment can be closed.
  
\item The \texttt{enumerate} environment creates ordered lists. The
  \texttt{itemize} environment creates unordered lists. The
  \texttt{table} environment is used to encapsulate table-like content.
  \marginpar{At the end of your content begin and end an environment by
    the name of \texttt{tiny}. Inside this environment type your
    favorite thing to shout. Use proper latex quotes (\textit{i.e.}\
    \`{} \`{} \ldots '' and not "\ldots"). Compile the document.}

\end{itemize}

\section{Error Debugging}

You will, invariably, make errors. However, you'll probably never be
the first person to make any particular error. The four most common
errors are
\begin{enumerate}
\item you type a command incorrectly
\item you forget to \texttt{usepackage} the package that supplies a
  macro which \LaTeX~interprets as if you typed a command incorrectly
\item you don't \texttt{end} what you \texttt{begin} or do so in the
  right order
\item you have un-balanced braces/parenthesis/brackets depending on
  what \LaTeX~is looking for.
\end{enumerate} \marginpar{Place the \texttt{tiny} environment inside a
  \texttt{center} environment. Now, change one of the \texttt{begin}
  commands to \texttt{benign}. Compile it. This is not benign! Look at
  the error message. Correct this mistake and swap the two
  \texttt{end} statements. Compile the document now. Look at the error
  message.}


\section{Solving Problems and Getting Help}
\begin{itemize}
\item Don't panic.
\item Check the usual suspects.
\item Google a description of your problem in addition to ``latex
  ctan.'' Including ``latex'' without ``ctan'' is a bad, bad, bad,
  bad, bad idea!
\item Comment out potential sources of the problem until you
  identify it. Slowly add things back in until you have removed only
  the offending portion. Can this be fixed?
\item Consult one of the \LaTeX~ books in the star lab.
\item Consult one of the many great \LaTeX~sites out
  there.\footnote{Google either ``latex ctan wikibook'' or ``latex
    comprehensive symbol''. These two are great.}
\item Ask the star lab fellow!
\end{itemize}


%% \bibliographystyle{}
%% \bibliography{}

% \end{document}

%%%%%%%%%%%%%%%%%%%%%%%%%%%%%%%%%%%%%%%%%%%%%%%%%%%%%%%%%%%%%%%%%%%%%%%%%%%%%%% 
%%%%%%%%%%%%%%%%%%%%%%%%%%%%%%%%%%%%%%%%%%%%%%%%%%%%%%%%%%%%%%%%%%%%%%%%%%%%%%% 
%%%%%%%%%%%%%%%%%%%%%%%%%%%%%%%%%%%%%%%%%%%%%%%%%%%%%%%%%%%%%%%%%%%%%%%%%%%%%%% 
