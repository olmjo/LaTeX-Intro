\part{Including Mathematics in \LaTeX{} Documents}

\section{Additional Packages}

\begin{itemize}

\item In order to avoid a lot of redundant and complicated code, it is
  helpful to load two particular packages each time you create a
  document with mathematics in it.

\item A bare-bones document requires only a \verb=\documentclass= line, followed
  by the \verb=\begin{document}= and \verb=\end{document}= lines.

\item We will now add \verb=\usepackage{amsmath, amsthm, amssym, amsfonts}= to
  the \textit{preamble} of the document. This tells \LaTeX{} to recognize
  certain commands defined in these packages.

\item We'll never use most of the functionality they add, but we'll
  often use some of it!

\item Create the following empty \LaTeX{} document in your text editor in which
  we'll practice mathematical commands.

\begin{center}
  \begin{minipage}{.8\linewidth}
    \begin{framed}
\begin{verbatim}
\documentclass{article}

\usepackages{amsmath, amsthm, amssymb, amsfonts}

% This is just a comment.

\begin{document}

\end{document}
\end{verbatim}
    \end{framed}
  \end{minipage}
\end{center}

\end{itemize}

\section{Math Modes}

\begin{itemize}

\item In order to display mathematical content in \LaTeX{}, we will use a
  new set of commands that we wouldn't need for, say, writing a
  letter. However, \LaTeX{} doesn't allow you to issue these commands
  just anywhere.

\item Rather, you need to be either in math mode or in a
  mathematical \textit{environment} where the math mode is implied.

\item There are two math modes which don't utilize special
  \textit{environments}: \textit{in-line} math and \textit{display
    math}.

\item These two are analogous to the way we include quotes in our own written
  work. \subparagraph{In-line} Sometimes the quotes are treated like any other
  text and printed in the same sized font without any special
  indentation. \subparagraph{Display} Sometimes the quotes begin on new lines,
  are indented, and quite distinct from the other text.

\item As a note, some commands and characters can only be entered in math
  mode. However, many, like the word ``computer'' can be entered in normal mode
  and math mode. Yet, if you think ``computer'' is different from
  ``$computer$'', then you must be aware which mode you use.

\item After this section, it will be assumed that all math command sequence are
  typed in math mode. Otherwise, they will almost always fail. {\large
    \textit{The math mode openings and closings are omitted!}}

\end{itemize}

\subsection*{In-line Math}
\begin{itemize}
\item The \textit{in-line} math mode is opened by typing a single dollar sign
  (\verb=$=) and then closed by typing a second single dollar sign.

\item The content is placed in between the two \verb=$= symbols.

\item When we type \verb!$55 / i = \pi^{-0.3}!, we get $55/i = \pi^{-0.3}$. Any
  lines immediately following the in-line math are entirely unaffected,
  including this one. And this one, too.
\end{itemize}

\subsection*{Display Math}

\begin{itemize}

\item The \textit{display} math mode is opened by typing \verb=\[= and closed by
  typing \verb=\]=. Another approach is to use double dollar signs as in the
  in-line mode \verb=$$= $\ldots$ \verb=$$=.

\item Again, the content to be displayed in the math mode is placed between the
  opening and closing key sequences.

\item Now, when we type \verb!\[ 55/i = \pi^{-0.3}\]!, we get
  \[55/i = \pi^{-0.3} \] which is the same as when we type
  \verb!$$ 55/i = \pi^{-0.3} $$! to get
  $$
  55/i = \pi^{-0.3}.
  $$

\item Clearly the appearance of the \textit{display} math is different from the
  \textit{in-line} math. However, sometimes the differences go beyond alignment
  and size. Both $\int_{0}^{\infty} \sum_{n=1}^{\infty} \frac{1}{n}~dx$ and
\[
\int_{0}^{\infty} \sum_{n=1}^{\infty} \frac{1}{n}~dx
\]
have the same math mode code, but their modes differ. In this example, the
limits of integration are the most obvious difference in how these expressions
are rendered
.
\end{itemize}

\section{Superscripts and Subscripts}

\begin{itemize}

\item To add a superscript to an expression use a hat, ``\verb=^='', after the
  term with the exponent. So, \verb=X^3= gives $X^3$.

\item If your exponent has more than one character to be rendered in the case of
  $X^{3+y}$, then we must surround the entire exponent in \textit{curly braces}
  (\verb={ }=).

  For this previous expression we'd type \verb=X^{3+y}=. In this sense, the
  \textit{curly braces} group terms. When in doubt, use curly braces to set the
  scope of your exponent. Had we omitted the curly braces, we'd see $X^3+y$
  which is quite different.

\item The principle for subscripting is identical, but we use the underscore,
  ``\verb=_='', instead. Typing \verb=X_{3+y}= gives $X_{3 + y}$.

\item Super- and sub-scripting can be recursive. However, one must be very
  careful with braces in this case. We can typeset $e^{x_{i}}$ with
  \verb=e^{x_{i}}=. Now, the braces around the \texttt{i} term are
  optional. Remove them, what happens to the rendering? Remove the braces around
  the $x_i$ term. Compile the document. What happens to the rendering?

\item Both superscripts and subscripts can be used simultaneously. In Social
  Choice, we often care about the median voter's preference, denoted
  $x_i^m$. This can be achieved with either \verb=x_i^m= or
  \verb=x_{i}^{m}=. The latter is more typing, but forces you to be explicit
  about the scope which tends to reduce the number of errors.

\end{itemize}

\section{Operators}

Here we show the symbols or commands to be typed in order to result in each of
the following symbols on the left. Again, as a reminder, these must all be in
math mode.

\begin{multicols}{3}
\begin{itemize}
\item[$+$] \texttt{+}
\item[$=$] \texttt{=}
\item[$-$] \texttt{-}
\item[$/$] \texttt{/}
\item[$<$] \texttt{<}
\item[$>$] \texttt{>}
\item[$\leq$] \texttt{\textbackslash leq}
\item[$\geq $] \texttt{\textbackslash geq}
\item[$\vee$] \texttt{\textbackslash vee}
\item[$\wedge$] \texttt{\textbackslash wedge}
\item[$\bigvee$] \texttt{\textbackslash bigvee}
\item[$\bigwedge$] \texttt{\textbackslash bigwedge}
\item[$\cup$] \texttt{\textbackslash cup}
\item[$\cap$] \texttt{\textbackslash cap}
\item[$\bigcup$] \texttt{\textbackslash bigcup}
\item[$\bigcap$] \texttt{\textbackslash bigcap}
\item[$\forall$] \texttt{\textbackslash forall}
\item[$\exists$] \texttt{\textbackslash exists}
\item[$\subset$] \texttt{\textbackslash subset}
\item[$\subseteq$] \texttt{\textbackslash subseteq}
\item[$\subsetneq$] \texttt{\textbackslash subsetneq}
\item[$\supset$] \texttt{\textbackslash supset}
\item[$\supseteq$] \texttt{\textbackslash supseteq}
\item[$\supsetneq$] \texttt{\textbackslash supsetneq}
\item[$\int$] \texttt{\textbackslash int}
\item[$\sum$] \texttt{\textbackslash sum}
\item[$\prod$] \texttt{\textbackslash prod}
\item[$\partial$] \texttt{\textbackslash partial}
\item[$\sim$] \texttt{\textbackslash sim}
\item[$\approx$] \texttt{\textbackslash approx}
\item[$\Leftrightarrow$] \texttt{\textbackslash Leftrightarrow}
\item[$\leftrightarrow$] \texttt{\textbackslash leftrightarrow}
\item[$\Leftarrow$] \texttt{\textbackslash Leftarrow}
\item[$\leftarrow$] \texttt{\textbackslash leftrightarrow}
\item[$\Rightarrow$] \texttt{\textbackslash Rightarrow}
\item[$\rightarrow$] \texttt{\textbackslash rightarrow}
\item[$\not ~$] \texttt{\textbackslash not}
\item[$\neg$] \texttt{\textbackslash neg}
\item[$\div$] \texttt{\textbackslash div}
\item[$\infty$] \texttt{\textbackslash infty}
\item[$\emptyset$] \texttt{\textbackslash emptyset}
\item[$\in$] \texttt{\textbackslash in}
\end{itemize}
\end{multicols}

\section{Greek Letters}

\begin{itemize}
\item Greek letters are uncomfortably common in some fields like statistics, so
  you will get used to including them in your documents.

\item Although there are neither $\zeta$ nor $\chi$ keys on your keyboard, we
  can include these symbols in the math environment in a very intuitive
  way. Type \verb=name-of-letter= or \verb=Name-Of-Letter= for the lower-case
  and upper-case versions of a Greek letter.

\item So, \verb=\delta= and \verb=\Delta= give $\delta$ and $\Delta$,
  respectively.

\item Not every letter has a distinct upper-case version provided. Although
  \verb=\beta= gives $\beta$ \verb=\Beta= is just a ``B''.

\item Not every letter has a special character, regardless of
  case. For example, the Greek omicron is just like the Roman `o', so
  \verb=o= gives $o$ which will be sufficient.

\item With Greek letters a particular issue about spaces often arises , although
  it isn't specific to Greek letters. \LaTeX{} commands do not need a leading
  space before the \verb=\=. But, they do need a trailing space so that \LaTeX{}
  knows the name of the command is over. So, \verb=X\beta= ($X\beta$) is
  rendered identically to \verb=X~\beta= ($X \beta$) because \LaTeX{} has its
  own way of interpreting whitespace.

  However, ``\verb=X\beta + \epsilon='' ($X\beta + \epsilon$) works whereas
  \verb=X\beta+ \epsilon= will not because ``\verb=\beta+='' is an undefined
  control sequence (i.e. we just made up that command).

\end{itemize}

\section{Other Letter-y Things}

Sometimes we need to present change the presentation of standard
letters or symbols in math mode. Here are several common examples. The
way these work is that whatever symbols are to have their face changed
are passed to these commands as arguments. Notice how the ``math''
faces remove white space

\begin{center}

  \begin{tabular}{c c c c}
    \hline \hline
    Modification & Command & Example & Common Uses\\
    \hline
    Normal Math Face & ---& $ABC XY$ & most maths\\
    Roman Face & \texttt{\textbackslash textrm\{\}} &  $\textrm{ABC XY}$& text within equation\\
    Bold Math Face & \texttt{\textbackslash mathbf\{\}} &  $\mathbf{ABC XY}$ & vectors and matrices\\
    Blackboard Math & \texttt{\textbackslash mathbb\{\}} &  $\mathbb{ABC XY}$ & special sets of numbers\\
    Calligraphic Math & \texttt{\textbackslash mathcal\{\}} & $\mathcal{ABC XY}$& arbitrary sets\\
    \hline \hline
  \end{tabular}

\end{center}

\section{Special Text-like Mathematical Expressions}

Certain functions, operators, and constructs pop up in math frequently which are
basically abbreviations for words. The cosine function---as in
$\cos(\pi)=1$---and the limit of an expression---as in $\lim_{x \rightarrow
  \infty} \frac{1}{x}$---are two such examples.

If we just typed \verb=$cos$= (yielding $cos$) or \verb=$lim$= (yielding $lim$) we get results
quite different. An incomplete list of the text-like expressions for which the
control sequences or commands are defined is below.

\begin{multicols}{4}
  \begin{itemize}
  \item[$\det$] \texttt{\textbackslash det}
  \item[$\lim$] \texttt{\textbackslash lim}
  \item[$\max$] \texttt{\textbackslash max}
  \item[$\min$] \texttt{\textbackslash min}
  \item[$\inf$] \texttt{\textbackslash inf}
  \item[$\sup$] \texttt{\textbackslash sup}
  \item[$\cos$] \texttt{\textbackslash cos}
  \item[$\sin$] \texttt{\textbackslash sin}
  \item[$\tan$] \texttt{\textbackslash tan}
  \item[$\exp$] \texttt{\textbackslash exp}
  \item[$\Pr$] \texttt{\textbackslash Pr}
  \item[$\arg$] \texttt{\textbackslash arg}
  \end{itemize}
\end{multicols}

\section{Delimiters}
Many times, we need to express groupings through the use of delimiters. For
example, $(100 - 100)^{100}$ is quite different than $100 - 100^{100}$. For
parentheses we can just enter ``\texttt{(x)}''. For curly braces we enter
``\texttt{\{x\}}''. And, lastly, for square brackets we use
``\texttt{[x]}'. These produce $(x)$, $\{x\}$, and $[x]$, respectively.

In longer expressions, though, the results can look funny.  In
\[
[\sum_{x=1}^{3} \{ (\int_{0}^{x} (e^{y}~dy))\}],
\]
the size of the delimiters is off. However, in
\[
\left[\sum_{x=1}^{3} \left\{ \left(\int_{0}^{x} (e^{y}~dy)\right)\right\}\right],
\]
the problem is gone.

If we type \texttt{\textbackslash left(}, \texttt{\textbackslash
  left\textbackslash\{}, or \texttt{\textbackslash left[} and close the
expression off with the approprate \texttt{\textbackslash right)},
\texttt{\textbackslash right\textbackslash\}}, or \texttt{\textbackslash
  right]}, the sizes are determined by the size of the internal expression as
opposed to being constant.

\section{Accents}

Often times we need to modify slightly a character to denote that the construct
has changed. For example, we want

\begin{itemize}
\item not the true parameter $\beta$ but the estimate $\hat{\beta}$,
\item not the random variable $y$, but the sample mean $\bar{y}$, or
\item not the singleton $x$, but a vector $\vec{x}$.
\end{itemize}

A partial list of these accents is below.
\begin{multicols}{3}
\begin{itemize}
\item[$\acute{a}$] \texttt{\textbackslash acute\{a\}}
\item[$\bar{b}$] \texttt{\textbackslash bar\{b\}}
\item[$\breve{c}$] \texttt{\textbackslash breve\{c\}}
\item[$\check{d}$] \texttt{\textbackslash check\{d\}}
\item[$\dot{e}$] \texttt{\textbackslash dot\{e\}}
\item[$\ddot{f}$] \texttt{\textbackslash ddot\{f\}}
\item[$\grave{g}$] \texttt{\textbackslash grave\{g\}}
\item[$\hat{h}$] \texttt{\textbackslash hat\{h\}}
\item[$\tilde{\imath}$] \texttt{\textbackslash tilde\{\textbackslash imath\}}
\item[$\vec{\jmath}$] \texttt{\textbackslash vec\{\textbackslash jmath\}}
\item[$\widehat{xyz}$] \texttt{\textbackslash widehat\{\textbackslash xyz\}}
\item[$\widetilde{xyz}$] \texttt{\textbackslash widetilde\{\textbackslash xyz\}}

\end{itemize}
\end{multicols}

\section{Assignment!}

\subsection*{Questions}

Write \LaTeX{} code for the following expressions. Confirm that your compiled
document matches these.

\begin{enumerate}
\item
\[
\neg\left(\left(T \vee F\right) \wedge F\right) \vee \left( T \right)
\]

\item
\[
e^{ix} = \cos x + i \sin x
\]

\item
\[
y_i = \beta_0 + \beta_1 x_i + \beta_2 x_i^2 + \epsilon_i
\]

\item
\[
\left(\sum_{x=1}^{\infty} (1/x) \right) - \left(\sum_{x=1}^{\infty}
  (1/x)^2 \right)
\]
\end{enumerate}

%%%%%%%%%%%%%%%%%%%%%%%%%%%%%%%%%%%%%%%%%%%%%%%%%%%%%%%%%%%%

\subsection*{Answers}

\begin{enumerate}

%%%%%%%%%%%%%%%%%%%%%%%%%%%%%%%%%%%%%%%%%%%%%%%%%%%%%%%%%%%%

\item ~\\

  \begin{turn}{180}
    \begin{minipage}{\linewidth}
      \begin{framed}
\begin{verbatim}

\[
\neg\left(\left(T \vee F\right)
\wedge F\right) \vee \left( T \right)
\]
\end{verbatim}
      \end{framed}
    \end{minipage}

  \end{turn}

%%%%%%%%%%%%%%%%%%%%%%%%%%%%%%%%%%%%%%%%%%%%%%%%%%%%%%%%%%%%

\item ~\\

  \begin{turn}{180}
    \begin{minipage}{\linewidth}
      \begin{framed}
\begin{verbatim}


\[
e^{ix} = \cos x + i \sin x
\]

\end{verbatim}
      \end{framed}
    \end{minipage}

  \end{turn}

%%%%%%%%%%%%%%%%%%%%%%%%%%%%%%%%%%%%%%%%%%%%%%%%%%%%%%%%%%%%

\item ~\\
  \begin{turn}{180}
    \begin{minipage}{\linewidth}
      \begin{framed}
\begin{verbatim}
\[
y_i = \beta_0 + \beta_1 x_i +
\beta_2 x_i^2 + \epsilon_i
\]
\end{verbatim}
      \end{framed}
    \end{minipage}
  \end{turn}

%%%%%%%%%%%%%%%%%%%%%%%%%%%%%%%%%%%%%%%%%%%%%%%%%%%%%%%%%%%%

\item ~\\
  \begin{turn}{180}
    \begin{minipage}{\linewidth}
      \begin{framed}
\begin{verbatim}
\[
\left(\sum_{x=1}^{\infty} (1/x) \right) -
\left(\sum_{x=1}^{\infty}
  (1/x)^2 \right)
\]
\end{verbatim}
      \end{framed}
    \end{minipage}
  \end{turn}

%%%%%%%%%%%%%%%%%%%%%%%%%%%%%%%%%%%%%%%%%%%%%%%%%%%%%%%%%%%%

\end{enumerate}


%%% Local Variables:
%%% mode: latex
%%% TeX-master: "../../tutorial"
%%% End: