\part{An Introduction to Using \LaTeX{}}
\section{What is \LaTeX{} (lay-teck) or LaTeX, but not latex or Latex?}

\begin{itemize}
  
\item \LaTeX{} is an open document preparation system used by many
  students, academics, and publishers whose content involves technical
  topics (\textit{e.g.}\ math professors, engineers, students of Kevin
  Clarke).
  
\item \LaTeX{} is a markup language which means that the content and
  structure are described by ``tags.'' If you go to the University of
  Rochester's website (\url{www.rochester.edu}) and view the HTML
  source for the main page it will look (incredibly) different from
  how that source code is rendered by your browser. \LaTeX{} is much
  more similar to HTML source than it is to a word processor
  (\textit{e.g.}  MS Word, OpenOffice Writer, Google Documents). When
  you work with a \LaTeX{} file, you describe how the file should
  look, but this is not displayed for you in real-time.
\end{itemize}
  
\section{Why Would You Use \LaTeX{}?}
\LaTeX{} has a number of distinct advantages:
\begin{enumerate}
\item A \LaTeX{} document is very pleasing to the eye without the need
  for any customization.
  
\item As a graduate student, you mainly produce content and are
  neither a publisher nor a graphic designer.  \LaTeX{} separates
  content from appearance.

\item \LaTeX{}'s ability to incorporate math is un-rivaled.

\item People in this discipline who do things like what you will do
  will expect it.

\item You will never get locked out from the fruits of your
  intellectual labor because of licenses or old software. The software
  is open and the files you create are ``flat text'' files. If someone
  has a computer, they can read your \LaTeX{} files.

\item The markup code for \LaTeX{} math is one of the canonical ways
  of representing math in plain-text environments (\textit{e.g.}\
  email, instant messaging).
\end{enumerate}

\section{How Does \LaTeX{} Work?}

\par The process of creating a \LaTeX{} document is quite different
from creating one in a word processor. Although most of the technical
details can safely be ignored, some understanding of the different
components comprising a functioning \LaTeX{} system is helpful and
very useful when asking for assistance.\footnote{This section is meant
  to be accompanied by the workflow diagram available at
  \url{http://www.rochester.edu/college/gradstudents/jolmsted/files/teaching/LaTeX/LaTeX_Workflow.svg}.
}

The steps to creating a document are:
\begin{enumerate}
\item Have an idea!
\item Create a plain text document containing the content in a text
  editor.
\item Submit the plain text document to the \LaTeX{} binaries.
\item \LaTeX{} pulls in any additional files that are asked for---but
  they have to be asked for---and creates, among other files, the
  document you will print or email.
\item View the result/rendering of your \LaTeX{} document.
\item Be happy.
\end{enumerate}

Notice that this is different from the word processor workflow. In
that case, Steps 2, 4, and 5 are basically integrated. Step 3 is
unnecessary. Step 6 may or may not happen.

\paragraph{Step 2} In the star lab, the Windows machines have the WinEdt
text editor. You could use any number of other text editors, but
WinEdt is nice because it integrates a lot of the other tools
seamlessly. We typically denote \LaTeX{} files with the \texttt{.tex}
file extension. Although it isn't necessary in some cases, most
software and operating systems will make your life easier if you are
willing to name \LaTeX{} files what they expect you to name them.

\paragraph{Step 3} You can achieve this either using the WinEdt
interface or issuing a command at the command prompt (in most cases,
this is no real choice). \TeX{} is different from \LaTeX{}. The
differences aren't important here, but know that the \TeX{} buttons
will likely not work.

\par Also, \LaTeX{}-ing your document can proceed through one of two
different branches. I will focus on the \texttt{pdflatex} branch and
not the \texttt{latex} branch. \texttt{pdflatex} will convert your
\texttt{.tex} file into a \texttt{.pdf} document. Included images must
be \texttt{.jpg}, \texttt{.png}, or \texttt{.pdf}
files. \texttt{latex} will convert your file into a \texttt{.dvi} file
which you'll eventually convert to a \texttt{.pdf}. Included images
must be \texttt{.eps} or \texttt{.ps} files. There are some advantages
to the pdf-less \texttt{latex} branch, I prefer the \texttt{pdflatex}
branch and we will walk through this one together.

\paragraph{Step 4} Sometimes various markup in the \LaTeX{} file will
require contributed packages which tell the \LaTeX{} binaries how the
content should be rendered. However, making sure these packages are
placed in the right spot on your system and up-to-date can be
involved. \LaTeX{} package managers simplify this. On the machines in
the star lab MikTeX is used. By and large, you can and should ignore
this fact in the beginning of your \LaTeX{} experience. What is
important is to realize the following:
\[
\textrm{\LaTeX{}} \neq \textrm{WinEdt} \neq \textrm{MikTeX}.
\]

\paragraph{Step 5} In the star lab, Adobe programs are available to view
(and edit) \texttt{.pdf} files. This can be achieved through the
WinEdt interface.

\paragraph{Step 6} No software necessary.

\section{A Simple Task}
\begin{enumerate}
\item Go to \url{www.rochester.edu/college/gradstudents/jolmsted/teaching/LaTeX/}.
\item Save the file \texttt{FakeFile.tex} to the desktop.
\item Open this file in WinEdt.
\item Run \texttt{pdflatex} on the \texttt{.tex} file and view it.
\item Run \texttt{latex} on the \texttt{.tex} file and view it.
\item Notice how many extra files this process creates.
\end{enumerate}

You have successfully created your first \LaTeX{} document and
simultaneously learned to always keep \LaTeX{} files/projects in
separate directories because the output will create clutter. Consider
this a windfall!


%% \bibliographystyle{}
%% \bibliography{}

% \end{document}

% %%%%%%%%%%%%%%%%%%%%%%%%%%%%%%%%%%%%%%%%%%%%%%%%%%%%%%%%%%%%%%%%%%%%%%%%%%%%%%% 
% %%%%%%%%%%%%%%%%%%%%%%%%%%%%%%%%%%%%%%%%%%%%%%%%%%%%%%%%%%%%%%%%%%%%%%%%%%%%%%% 
% %%%%%%%%%%%%%%%%%%%%%%%%%%%%%%%%%%%%%%%%%%%%%%%%%%%%%%%%%%%%%%%%%%%%%%%%%%%%%%% 


% %%% Local Variables: 
% %%% mode: latex
% %%% TeX-master: t
% %%% End: 
